\documentclass[10pt,a4paper]{article}
\usepackage[T1]{fontenc}
\usepackage[utf8]{inputenc}
\usepackage{graphicx}
\usepackage[polish]{babel}
\usepackage{hyperref}
\usepackage[paper=a4paper,margin=1in]{geometry}
\usepackage{tabularx} 
\usepackage{helvet}
\usepackage{amsfonts}
\usepackage{amsthm}
\usepackage{mathtools}
\usepackage{algpseudocode}
\usepackage{tikz}
\usepackage{tkz-graph}
\usepackage{listings,xcolor}
\renewcommand{\familydefault}{\sfdefault}
\setlength{\parindent}{0pt}
\newtheorem{theorem}{Twierdzenie}
\newtheorem{lemma}{Lemat}
\newtheorem{conculsion}{Wniosek}

\begin{document}
	
	\section{Problem}
	
	\textbf{Temat projektu: GOLF} \\~\\
	Na polu golfowym znajduje się $n$ piłeczek oraz $n$ dołków na piłeczki. Golfiści chcą jednocześnie umieścić każdy swoją piłeczkę w którymś z dołków. W tym celu ustalają między sobą, który celuje do którego dołka, ale w taki sposób, by tory ich	piłeczek się nie przecinały, co gwarantuje brak zderzeń piłeczek.\\~\\ 
	Załóżmy, że piłeczki i dołki są punktami na płaszczyźnie oraz żadne trzy z tych punktów nie są współliniowe, a tory piłeczek są odcinkami prostej. Przedstaw działający w czasie $O(n^2\log n)$ algorytm przydzielania piłeczek do dołków tak, aby żadne dwa tory piłeczek się nie przecinały.\\~\\
	\textit{Wskazówka:} W czasie $O(n\log n)$ można znaleźć prostą przechodzącą przez jedną piłeczkę i jeden dołek taką, że liczby piłeczek i dołków po tej samej stronie tej prostej są równe.
	
	\section{Sposób rozwiązania}
	
	Algorytm bazuje na strategii dziel i zwyciężaj. Dla zadanego zbioru wyszukujemy uderzenie $S$, po którego lewej lub prawej stronie (nie ma to znaczenia ze względu na równoliczność zbiorów $\mathbb{B}$ oraz $\mathbb{H}$) znajduje się dokładnie tyle samo piłeczek, co dołków. Następnie wywołujemy ten algorytm osobno dla piłeczek i dołków znajdujących się po lewej stronie, a osobno po prawej. Wywołując ten algorytm dla jednej piłeczki i jednego dołka, zwracamy natychmiast uderzenie składające się z tej piłeczki i tego dołka. Wszystkie znalezione przez ten algorytm uderzenia tworzą rozwiązanie zadania. \\~\\
	Aby znaleźć uderzenie $S$, po którego jednej ze stron znajduje się dokładnie tyle samo piłeczek, co dołków, zastosujemy algorytm podobny w budowie do algorytmu Grahama znajdowania otoczki wypukłej punktów w przestrzeni dwuwymiarowej. Wyszukujemy punkt o najmniejszej współrzędnej rzędnej (jeżeli jest ich kilka, to wybieramy ten o najmniejszej współrzędnej odciętej) $O$ i przesuwamy początek układu współrzędnych do tego punktu. Następnie sortujemy pozostałe punkty po kącie między wektorem od początku układu do tego punktu a osią $OX$ (ponieważ dowolne trzy punkty z obu zbiorów nie są współliniowe, to każdy punkt będzie miał inną wartość tego kąta). \\~\\
	Każdemu punktowi, w zależności od tego, czy jest piłeczką czy dołkiem, przyporządkowana jest wartość liczbowa: dla piłeczki jest to $1$, zaś dla dołka $-1$. Inicjalizujemy zmienną pomocniczą wartością dla punktu $O$ i następnie przechodzimy po posortowanej liście punktów, dodając do tej zmiennej wartość aktualnego punktu. W momencie, gdy zmienna pomocnicza osiągnie wartość $0$, algorytm kończy działanie, a ostatnio rozpatrywany punkt tworzy z punktem $O$ uderzenie, które spełnia wspomniany warunek.
	
\end{document}