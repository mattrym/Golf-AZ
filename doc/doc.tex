\documentclass[10pt,a4paper]{article}
\usepackage[T1]{fontenc}
\usepackage[utf8]{inputenc}
\usepackage{amssymb}
\usepackage{graphicx}
\usepackage[polish]{babel}
\usepackage{hyperref}
\usepackage[paper=a4paper,margin=1in]{geometry}
\usepackage{tabularx} 
\usepackage{helvet}
\usepackage{amsfonts}
\usepackage{amsthm}
\usepackage{mathtools}
\usepackage{algpseudocode}
\usepackage{tikz}
\usepackage{tkz-graph}
\usepackage{listings,xcolor}
\usepackage{scrextend}
\usepackage{float}
\usepackage{subcaption}
\renewcommand{\familydefault}{\sfdefault}
\setlength{\parindent}{0pt}
\newtheorem{theorem}{Twierdzenie}
\newtheorem{lemma}{Lemat}
\newtheorem{invariant}{Niezmiennik}
\newtheorem{conculsion}{Wniosek}

\begin{document}
	\begin{titlepage}
		\newgeometry{top=1in,bottom=1in,right=1.5in,left=1.5in}
		\begin{center}
			{\fontsize{14}{12}\selectfont Politechnika Warszawska \\ Wydział Matematyki i Nauk Informacyjnych}
			
		\end{center}
		
		\vspace{1cm}
		\begin{center}
			\includegraphics[width=0.3\textwidth]{images/logo.png}
		\end{center}
		\vspace{3cm}
		
		\begin{center}
			\textbf{{\fontsize{26}{12}\selectfont Algorytmy Zaawansowane}}
			
			\vspace{2cm}
			\textbf{{\fontsize{22}{12}\selectfont Dokumentacja projektowa}}
			\vspace{1cm}
			
			\textbf{{\fontsize{13.5}{12}\selectfont Piotr Piwowarski, Mateusz Rymuszka}}
			
			\vspace{6cm}
			\textbf{{\fontsize{13.5}{12}\selectfont \today}}
		\end{center}  
	\end{titlepage}
	
	{\fontsize{13.5}{12}\selectfont
		\tableofcontents
		\vspace{1cm}
		{\renewcommand{\arraystretch}{2.0}
		
	}}
	
	\newpage
	
	\section{Słownik pojęć}
	
	W niniejszej dokumentacji wykorzystane są poniższe pojęcia:
	\begin{itemize}
		\item \textbf{piłeczka} - punkt na płaszczyźnie dwuwymiarowej ze zbioru $\mathbf{B}$;
		\item \textbf{dołek} - punkt na płaszczyźnie dwuwymiarowej ze zbioru $\mathbf{H}$;
		\item \textbf{uderzenie} - odcinek łączący piłeczkę i dołek na płaszczyźnie dwuwymiarowej;
		\item \textbf{uderzenie zbalansowane} - uderzenie, w którym po każdej ze stron półpłaszczyzn rozgraniczonych przez prostą zawierającą te uderzenie znajduje się tyle samo piłeczek, co dołków;
		\item \textbf{dopasowanie} (piłeczek $\mathbf{B}$ do dołków $\mathbf{H}$) - zbiór uderzeń pokrywających zbiory punktów $\mathbf{B}$ i $\mathbf{H}$;
		\item \textbf{planarne dopasowanie} - dopasowanie, w którym żadna para uderzeń nie ma punktu wspólnego;
	\end{itemize}
	
	\section{Problem}
	
	\textbf{Temat projektu: GOLF} \\~\\
	Na polu golfowym znajduje się $n$ piłeczek oraz $n$ dołków na piłeczki. Golfiści chcą jednocześnie umieścić każdy swoją piłeczkę w którymś z dołków. W tym celu ustalają między sobą, który celuje do którego dołka, ale w taki sposób, by tory ich	piłeczek się nie przecinały, co gwarantuje brak zderzeń piłeczek.\\~\\ 
	Załóżmy, że piłeczki i dołki są punktami na płaszczyźnie oraz żadne trzy z tych punktów nie są współliniowe, a tory piłeczek są odcinkami prostej. Przedstaw działający w czasie $O(n^2\log n)$ algorytm przydzielania piłeczek do dołków tak, aby żadne dwa tory piłeczek się nie przecinały.\\~\\
	\textit{Wskazówka:} W czasie $O(n\log n)$ można znaleźć prostą przechodzącą przez jedną piłeczkę i jeden dołek taką, że liczby piłeczek i dołków po tej samej stronie tej prostej są równe.
	
	\section{Zadanie oraz jego rozmiar}
	
	\subsection{Zadanie}
	Dwa zbiory $\mathbf{B}$ oraz $\mathbf{H}$ różnych, trójkami niewspółliniowych punktów o liczności $n$ każdy z przestrzeni $\mathbb{R}^{2}$ (reprezentujących odpowiednio położenia piłeczek oraz położenia dołków).
	
	\subsection{Rozmiar zadania}
	
	Całkowita liczba punktów wynosi $2n$. Biorąc jednak pod uwagę fakt, że rozpatrujemy pary piłka - dołek oraz dla prostoty dalszej analizy możemy przyjąć. że rozmiar zadania nie zależy od stałej i wynosi $n$.
	
	\section{Sposób rozwiązania}
	
	\subsection{Opis}
	
	Algorytm bazuje na strategii dziel i zwyciężaj. Dla zadanego zbioru wyszukujemy uderzenie zbalansowane, tj. takie, po którego lewej lub prawej stronie (nie ma to znaczenia ze względu na równoliczność zbiorów $\mathbf{B}$ oraz $\mathbf{H}$) znajduje się dokładnie tyle samo piłeczek, co dołków. Następnie wywołujemy ten algorytm osobno dla piłeczek i dołków znajdujących się po lewej stronie, a osobno po prawej. \\~\\
	Wywołując algorytm dla pustych zbiorów piłeczek i dołków, otrzymamy zbiór pusty. Wywołując algorytm dla jednej piłeczki i jednego dołka, zwracamy uderzenie składające się z tej piłeczki i tego dołka. Wszystkie znalezione przez ten algorytm uderzenia tworzą rozwiązanie zadania. \\~\\
	Aby znaleźć uderzenie zbalansowane zastosujemy algorytm o podobnym schemacie do algorytmu Grahama znajdowania otoczki wypukłej punktów w przestrzeni dwuwymiarowej:
	
	\begin{enumerate}
		\item Wyszukujemy punkt o najmniejszej współrzędnej rzędnej (jeżeli jest ich kilka, to wybieramy ten o najmniejszej współrzędnej odciętej) $L$ i przesuwamy początek układu współrzędnych do tego punktu (przesuwamy wszystkie punkty o wektor $\overrightarrow{LO}$).
		\item Sortujemy pozostałe punkty po kącie między wektorem $OP$ od początku układu do tego punktu a osią $OX$. Ponieważ dowolne trzy punkty z obu zbiorów nie są współliniowe, to każdy punkt będzie miał inną wartość tego kąta, za wyjątkiem punktu $L$ - w tym celu można np. wprowadzić sortowanie leksykograficzne z sortowaniem po odległości od $O$ jako drugi predykat.
		\item Każdemu punktowi, w zależności od tego, czy jest piłeczką czy dołkiem, przyporządkowana jest wartość liczbowa: dla piłeczki jest to $1$, zaś dla dołka $-1$. Inicjalizujemy zmienną pomocniczą wartością zerową i następnie przechodzimy po posortowanej liście punktów, dodając do tej zmiennej wartość aktualnego punktu.\\~\\
		W momencie, gdy zmienna pomocnicza osiągnie ponownie wartość $0$, algorytm kończy działanie, a ostatnio rozpatrywany punkt tworzy z punktem $O$ uderzenie, które spełnia wspomniany warunek.
	\end{enumerate}

	Po znalezieniu uderzenia zbalansowanego oraz znalezieniu rozwiązań dla zmniejszych zadań na lewo i na prawo od tego uderzenia, jako wynik zwracamy sumę teoriomnogościową rozwiązania dla punktów po lewej, zbiór jednoelementowy zawierający znalezione uderzenie zbalansowane oraz rozwiązanie dla punktów po prawej.
	
	\subsection{Pseudokod}

	\subsubsection{Pseudokod ogólnego algorytmu}

	\begin{algorithmic}
		\State \textbf{Wejście:} zbiór piłeczek $\mathbf{B}$ oraz dołków $\mathbf{H}$  \\
		
		\Function {FindPlanarMatching}{$\mathbf{B}$, $\mathbf{H}$}\\
			\If{$|\mathbf{B}| = |\mathbf{H}| = 0$} \Comment{(1)}
				\State \Return $\emptyset$ 
			\EndIf
			\\
			\If{$|\mathbf{B}| = |\mathbf{H}| = 1$} \Comment{(1)}
				\State $B \coloneqq $ punkt z $\mathbf{B}$
				\State $H \coloneqq $ punkt z $\mathbf{H}$
				\State \Return $\left\{\overline{BH}\right\}$ 
			\EndIf
			\\
			\State $\overline{LS} \coloneqq$ \Call{FindBalancedMatch}{$\mathbf{B}$, $\mathbf{H}$} \Comment{(2)}
			\\
			\State $\mathbf{B}_{1} \coloneqq $ punkty z $\mathbf{B}$ na lewo od $\overrightarrow{LS}$ \Comment{(3)}
			\State $\mathbf{H}_{1} \coloneqq $ punkty z $\mathbf{H}$ na lewo od $\overrightarrow{LS}$ \Comment{(3)}
			\State $\mathbf{B}_{2} \coloneqq $ punkty z $\mathbf{B}$ na prawo od $\overrightarrow{LS}$ \Comment{(3)}
			\State $\mathbf{H}_{2} \coloneqq $ punkty z $\mathbf{H}$ na prawo od $\overrightarrow{LS}$ \Comment{(3)}
			\\
			\State $\mathbf{M}_{1} \coloneqq $ \Call{FindPlanarMatching}{$\mathbf{B}_{1}$, $\mathbf{H}_{1}$}
			\State $\mathbf{M}_{2} \coloneqq $ \Call{FindPlanarMatching}{$\mathbf{B}_{2}$, $\mathbf{H}_{2}$} 
			\State \Return $\mathbf{M}_{1} \cup \left\{\overline{LS}\right\} \cup \mathbf{M}_{2}$ \Comment{(4)} \\ 
		\EndFunction
	\end{algorithmic}

\pagebreak

	\subsubsection{Pseudokod algorytmu znajdującego uderzenie zbalansowane}
	
	\begin{algorithmic}
		\State \textbf{Wejście:} zbiór piłeczek $\mathbf{B}$ oraz dołków $\mathbf{H}$  \\
		
		\Function {FindBalancedMatch}{$\mathbf{B}$, $\mathbf{H}$}\\
			\State $\mathbf{P} \coloneqq \mathbf{B} \cup \mathbf{H}$ \Comment{(1)}
			\State $O \coloneqq$ początek układu współrzędnych
			\State $L \coloneqq$ najbardziej położony na lewo spośród najniższych punktów z $\mathbf{P}$ \Comment{(2)}
			\\
			\State przesuń wszystkie punkty z $\mathbf{P}$ o wektor $\overrightarrow{LO}$ \Comment{(3)}
			\State posortuj $\mathbf{P}$ leksykograficznie po kącie między wektorem $\overrightarrow{OP}$ a osią $OX$ i odległości od $O$ \Comment{(4)} 
			\\
			\State balance $\coloneqq 0$
			\ForAll{$P \in \mathbf{P}$} \Comment{(5)} 
				\If{$P \in \mathbf{B}$}
					\State balance $\mathrel{{+}{=}} 1$
				\Else
					\State balance $\mathrel{{-}{=}} 1$
				\EndIf
				\If{balance $= 0$}
					\State \Return $\overline{LP}$
				\EndIf
			\EndFor\\
		\EndFunction
	\end{algorithmic}
	
	\section{Rozwiązanie przykładowego zadania}
	
	Załóżmy, że mamy dwa równoliczne zbiory i każdemu z elementów zbioru odpowiada pewien punkt na płaszczyźnie:
	
	\begin{itemize}
		\item Zbiór piłek $\mathbf{B} = \{B1, B2, B3, B4\}$ 
		\item Zbiór dołków $\mathbf{H} = \{H1, H2, H3, H4\}$ 
	\end{itemize}

	\begin{figure}[H]
		\centering
 			\includegraphics[scale=0.5]{images/step1.png}
 		 	\caption{Początkowy układ dołków i piłeczek.}
 		\label{fig:step1}
	\end{figure}
	
	Pierwszym krokiem jest wybranie punktu, który ma najmniejszą wartość współrzędnej Y, czyli $B1$. Następnie, po przesunięcia środka układu współrzędnych w jego miejsce, sortujemy pozostałe punkty tak jak jest w opisie algorytmu. 
	
	\begin{figure}[H]
		\centering
 			\includegraphics[scale=0.5]{images/step2.png}
 			\caption{Znalezienie pierwszego punktu dzielącego płaszczyznę.}
 		\label{fig:step1}
	\end{figure}
	
	Następnie przechodzimy po kolei po posortowanych punktach, odpowiednio zwiększając lub zmniejszając wartość balansu. W momencie dotarcia do punktu $H1$ wartość balansu jest równa $0$, dlatego zapamiętujemy uderzenie $\{B1,H1\}$, a parę tych punktów możemy pominąć w dalszych poszukiwaniach.\\~\\
	Punkty, które zostaly znalezione do momentu znalezienia punktu dzielącego płaszczyznę kwalifikujemy do osobnego podproblemu (Grupa 1), podobnie jak punkty jeszcze nieodwiedzone (Grupa 2).
	
	\begin{figure}[H]
		\centering
 			\includegraphics[scale=0.5]{images/step3.png}
 			\caption{Podział na podproblemy.}
 		\label{fig:step1}
	\end{figure}		

	Postępując analogicznie w powstałych podproblemach, znajdujemy kolejne pary punktów, które dodajemy do wynikowego zbioru uderzeń.

	\begin{figure}[H]
		\centering
		\begin{subfigure}[b]{0.4\linewidth}
			\includegraphics[scale=0.5]{images/step41.png}
			\caption{Grupa 1.}
		\end{subfigure}
		\begin{subfigure}[b]{0.4\linewidth}
			\includegraphics[width=\linewidth]{images/step421.png}
			\caption{Grupa 2.}
		\end{subfigure}
		\caption{Rozwiązywanie podproblemów.}
		\label{fig:coffee}
	\end{figure}

	\begin{figure}[H]
		\centering
		\begin{subfigure}[b]{0.4\linewidth}
			\includegraphics[scale=0.6]{images/step422.png}
			\caption{Stworzenie kolejnego podproblemu.}
		\end{subfigure}
		\begin{subfigure}[b]{0.4\linewidth}
			\includegraphics[scale=0.7]{images/step423.png}
			\caption{Grupa 3.}
		\end{subfigure}
		\caption{Rozwiązywanie podproblemów}
		\label{fig:coffee}
	\end{figure}
	
	Gdy nie pozostały żadne nierozwiązane podproblemy (wszystkie punkty zostały dobrane w pary), możemy zwrócić ostateczny zbiór uderzeń $\mathbf{U} = \{\{B1,H1\},\{B2,H2\},\{B3,H3\},\{B4,H4\}\}$ będący wynikiem działania algorytmu.
		
	\section{Znalezienie punktu dzielącego płaszczyznę.}
	
	\subsection{Wejście}
	

	Plik wejściowy posiada następujące informacje:
	\begin{itemize}
		\item liczbę piłeczek lub liczbę dołków, czyli $n$ w pierwszej linii;
		\item współrzędne piłeczek rozdzielone przecinkami, każda piłeczka jest opisana w nowej linii;
		\item współrzędne dołków rozdzielone przecinkami, każdy dołek jest opisany w nowej linii;
	\end{itemize}

	\textbf{Przykład:} 
	
	\begin{verbatim}
		4
		0.1,0.1
		0.8,0.2
		1.0,1.8
		0.0,1.4
		0.9,0.8
		1.5,0.7
		0.5,1.7
		1.6,1.6
	\end{verbatim}
	
	\subsection{Wyjście}
	
	Plik wyjściowy posiada następujące informacje:
	\begin{itemize}
		\item indeks piłeczki w pliku wejściowym oraz indeks dołka w pliku wejściowym, które tworzą uderzenie, rozdzielone spacją; każde uderzenie jest opisane w nowej linii;
	\end{itemize}
	
	\textbf{Przykład:} 
	
	\begin{verbatim}
		1,1
		2,2
		3,3
		4,4
	\end{verbatim}
	
	\pagebreak
	
	\section{Dowód poprawności rozwiązania}
	
	\begin{lemma}\label{lemma:1}
		Niech $\mathbf{B}$ będzie zbiorem piłeczek, zaś $\mathbf{H}$ zbiorem dołków oraz niech $|\mathbf{B}| = |\mathbf{H}|$ . Jeżeli $\overline{LS}$ jest uderzeniem takim, że po jego lewej (prawej) stronie znajduje się tyle samo piłeczek co dołków, to dopasowanie będącego sumą teoriomnogościową:
		\begin{itemize}
			\item planarnego dopasowania $\mathbf{M}_{1}$ piłeczek na lewo od $\overrightarrow{LS}$ z dołkami na lewo od $\overrightarrow{LS}$
			\item zbioru jednoelementowego $\left\{\overline{LS}\right\}$
			\item planarnego dopasowania $\mathbf{M}_{2}$ piłeczek na prawo od $\overrightarrow{LS}$ z dołkami na prawo od $\overrightarrow{LS}$
		\end{itemize}
		jest planarne.
	\end{lemma}

	\begin{proof}
		Dla przypadku dla $n = 1$  teza jest oczywista. \\~\\
		Zauważmy, że jeżeli $LS$ spełnia założenie lematu, to z równoliczności $\mathbf{B}$ i $\mathbf{H}$ wynika, że jeżeli po jednej stronie uderzenia jest tyle samo piłeczek, co dołków, to po drugiej stronie również jest tyle samo piłeczek, co dołków (gdyż sumarycznie również musi być tyle samo piłeczek, co dołków). \\~\\
		Niech $\mathbf{B}_{1}$ i $\mathbf{H}_{1}$ będą zbiorami odpowiednio piłeczek i dołków na lewo od $\overrightarrow{LS}$, zaś $\mathbf{B}_{2}$ i $\mathbf{H}_{2}$ będą zbiorami piłeczek i dołków na lewo od $\overrightarrow{LS}$. Ponieważ $|\mathbf{B}_{1}| = |\mathbf{H}_{1}|$ i $|\mathbf{B}_{2}| = |\mathbf{H}_{2}|$, to pary $\mathbf{B}_{1}$ i $\mathbf{H}_{1}$ oraz $\mathbf{B}_{2}$ i $\mathbf{H}_{2}$ również są zadaniami należącymi do zbioru zadań rozpatrywanego problemu. \\~\\
		Zakładając, że dopasowanie $\mathbf{M}_{1}$ jest planarne wiemy, że dla dowolnej pary uderzeń z $\mathbf{M}_{1}$ te uderzenia nie mają punktu wspólnego. Dodatkowo, skoro $\mathbf{M}_{1}$ to zbiór uderzeń piłeczek znajdujących się na lewo od $\overrightarrow{LS}$ do dołków znajdujących się na lewo od $\overrightarrow{LS}$, to dowolne uderzenie $\mathbf{M}_{1}$ nie ma punktu wspólnego z $\overline{LS}$. Analogicznie, dowolna para uderzeń z $\mathbf{M}_{2}$ nie ma punktu wspólnego oraz żadne uderzenie z $\mathbf{M}_{2}$ nie ma punktu wspólnego z $\overline{LS}$. \\
		Jak łatwo zauważyć, w zbiorze $\mathbf{M}_{1} \cup \left\{\overline{LS}\right\} \cup \mathbf{M}_{2}$ są same uderzenia. Ponadto, dowolna para uderzeń z tego zbioru nie ma punktu wspólnego oraz te uderzenia pokrywają zbiór punktów $\mathbf{B} \cup \mathbf{H}$. Z tego wprost wynika, że $\mathbf{M}_{1} \cup \left\{\overline{LS}\right\} \cup \mathbf{M}_{2}$ jest dopasowaniem planarnym.
	\end{proof}

	\vspace{2em}

	\begin{theorem} \label{theorem:1}
		Algorytm znajdujący uderzenie zbalansowane znajduje takie dopasowanie dla dowolnych skończonych i równolicznych zbiorów $\mathbf{B}$ piłeczek i $\mathbf{H}$ dołków.
	\end{theorem}

	\begin{proof}
		Dla $n = 0$ oraz $n = 1$ dowód wynika wprost z definicji uderzenia zbalansowanego. \\~\\
		Operacje początkowe mają na celu przygotować punkty do wykonania właściwej części algorytmu. Wynikające z nich własności pomogą nam w udowodnieniu niezmienników głównej pętli. Załóżmy, że:
		\begin{enumerate}
			\item znaleźliśmy najniższy, najbardziej na lewo punkt $L$ spośród wszystkich piłeczek i dołków, 
			\item przesunęliśmy do punktu $L$ początek układu współrzędnych $O$
			\item posortowaliśmy punkty z $\mathbf{P}$ po kącie między wektorem $\overrightarrow{OP}$ a osią $OX$ oraz odległości od $O$
		\end{enumerate}
		Ostatni punkt nie wymaga sortowania wszystkich punktów po odległości od $O$, gdyż z założeń żadne trzy punkty z nie są współliniowe. Wymusza to jedynie, aby punkt $O = L$ był pierwszy w kolejności. \\~\\
		W celu dowiedzenia poprawności naszego algorytmu, pokażemy, że zachodzą pewne niezmienniki pętli głównej naszego algorytmu.

		\begin{addmargin}[1.25em]{0em}
			\begin{invariant} \label{invariant:1}
				Algorytm nie zakończy się w pierwszym kroku pętli.
			\end{invariant}
			
			\begin{proof}
				Dość oczywisty - w tym kroku zmienna \texttt{balance} będzie wynosić $-1$ lub $1$, więc algorytm nie przerwie pętli i będzie się wykonywał dalej. 
			\end{proof}
			
			\vspace{1em}
			
			\begin{invariant} \label{invariant:2}
				Algorytm zakończy się w którymś z kroków pętli.
			\end{invariant}
			
			\begin{proof}
				Zmienna \texttt{balance} jest zainicjowana zerem przed wejściem do pętli. Następnie, w każdym kroku pętli jest ona zwiększana o jeden (gdy punkt rozpatrywany w iteracji jest piłeczką) lub zmniejszana od jeden (gdy punkt rozpatrywany w iteracji jest dołkiem). \\~\\
				Gdyby algorytm nie zakończył się w którejś z pętli, oznaczałoby to, że zmienna \texttt{balance} nie przyjęła wartości zerowej po żadnym z kroków pętli, w szczególności - ostatnim. Zauważmy jednak, iż po ostatnim kroku pętli rozpatrzyliśmy wszystkie piłeczki i wszystkie dołki, których jest tyle samo. Oznacza to, że do zmiennej \texttt{balance} dodaliśmy dokładnie tyle, co od niej odjęliśmy. Po wykonaniu pętli powinna więc wynosić tyle, ile jej wartość początkowa, czyli $0$. Sprzeczność - algorytm zakończy się najpóźniej w ostatnim kroku pętli. 
			\end{proof}
			
			\vspace{1em}
			
			\begin{invariant} \label{invariant:3}
				W $k$-tym kroku pętli (gdzie $k \geq 3)$ na prawo (zgodnie z ruchem wskazówek zegara) od odcinka $\overline{OP}$ znajdują się punkty rozpatrywane w krokach od drugiego do $k-1$-tego.
			\end{invariant}
			
			\begin{proof}
				Punkty są posortowane rosnąco po kącie między wektorem $\overrightarrow{OP}$ a osią $OX$.  Wobec tego, dla różnych punktów $P_{i}$ oraz $P_{j}$, dla których te kąty wynoszą odpowiednio $\alpha_{i}$ oraz $\alpha_{j}$, jeżeli $i < j$, to $\alpha_{i} < \alpha_{j}$. \\~\\
				Wobec tego, kąt między wektorem $\overrightarrow{OP_{ji}}$ oraz $\overrightarrow{OP_{j}}$ wynosi $\alpha_{i} - \alpha_{j} < 0$, zatem punkt $P_{j}$ jest położony zgodnie z ruchem wskazówek zegara w stosunku do wektora $\overrightarrow{OP_{i}}$, co należało dowieść.
			\end{proof}
		
			\vspace{1em}
			
			\begin{invariant} \label{invariant:4}
				Zmienna \texttt{balance} po wykonaniu $k$-tego kroku pętli (poza pierwszym krokiem pętli) dla punktu $P$ wskazuje na różnicę między liczbą piłeczek a liczbą dołków na półpłaszczyźnie na prawo od prostej zawierającej wektor $\overrightarrow{OP}$, włącznie z tą prostą.
			\end{invariant}
			
			\begin{proof}[Dowód (indukcyjny)]
				Po pierwszym kroku pętli zmienna \texttt{balance} wynosi $-1$, gdy najniższy, najbardziej na lewo punkt jest dołkiem, albo $1$, gdy ten punkt jest piłeczką. Nie jest to istotne ze względu na niezmiennik \ref{invariant:1} mówiący, że algorytm nie zakończy się w pierwszym kroku działania.
				\begin{enumerate}
					\item Dla $k = 2$ nie ma punktów na prawo od wektora $\overrightarrow{OP}$. Zmienna \texttt{balance} jest wówczas sumą wartości dla pierwszego i drugiego punktu. Może wynieść:
					\begin{enumerate}
						\item $2$, gdy oba punkty są piłeczkami,
						\item $0$, gdy jeden z punktów jest piłeczką, a drugi dołkiem,
						\item $-2$, gdy oba punkty są dołkami.
					\end{enumerate}
					\item Założmy, że dla pewnego $k < 2n$ spełniona jest teza niezmiennika. Wówczas rozpatrzmy krok $(k+1)$-ty, w którym rozpatrujemy punkt $P$. \\~\\
					Na mocy niezmiennika \ref{invariant:3} na prawo od wektora $\overrightarrow{OP}$ znajdują się punkty rozpatrywane w krokach od drugiego do $k$-tego. Z kolei na mocy założenia indukcyjnego, zmienna \texttt{balance} przed wykonaniem $(k+1)$-go kroku wskazuje na różnicę między liczbą piłeczek a liczbą dołków dla punktów rozpatrywanych w poprzednich krokach algorytmu. \\~\\
					Zmienna \texttt{balance} zwiększy się o $1$, gdy rozpatrywany punkt $P$ będzie piłeczką, lub zmniejszy się o $1$, gdy będzie on dołkiem. Tak samo, różnica między liczbą piłeczek a dołków również zwiększy się o $1$, gdy rozpatrywany punkt $P$ będzie piłeczką, lub zmniejszy się o $1$, gdy będzie on dołkiem. Wobec tego, po wykonaniu $(k+1)$-tego kroku zmienna \texttt{balance} nadal będzie wskazywać na różnicę między liczbą piłeczek a liczbą dołków dla punktów rozpatrywanych w krokach od pierwszego do $(k+1)$-tego. 
				\end{enumerate}
			\end{proof}
			
			\vspace{2.5em}
			
			\begin{invariant} \label{invariant:5}
				Algorytm znajduje odcinek będący uderzeniem.
			\end{invariant}
			
			\begin{proof}
				Założmy, że znaleziony przez algorytm odcinek nie jest uderzeniem. Wobec tego, w pierwszym i ostatnim kroku algorytmu wartość \texttt{balance} albo się zwiększyła, albo się zmniejszyła. Dla ustalenia uwagi przyjmijmy, że w tych krokach rozpatrywano piłeczki i wartości zmiennej \texttt{balance} była zwiększana (drugi przypadek jest symetryczny).\\~\\
				Po pierwszym kroku wartość zmiennej \texttt{balance} wynosiła $1$. Z kolei, jeżeli po ostatnim kroku zmienna \texttt{balance} wynosiła $0$, to przed nim musiała wynosić $-1$ (w tym kroku rozpatrywaliśmy piłeczkę). W związku z tym, w krokach od drugiego do przedostatniego wartość zmiennej \texttt{balance} zmieniała się od $1$ do $-1$. \\~\\
				Ponieważ w jednym kroku zmienna \texttt{balance} może się mniejszyć maksymalnie o $1$, to po pewnym kroku między drugim a przedostatnim wartość zmiennej \texttt{balance} wynosiła $0$. Sprzeczność - algorytm powinien zakończyć się w momencie, gdy zmienna \texttt{balance} po raz pierwszy osiągnie wartość zerową.
			\end{proof}
		\end{addmargin}

		Na podstawie tych niezmienników wiemy, że algorytm zakończy się w którymś z kroków (poza pierwszym). Wiemy także, że po tym kroku zmienna \texttt{balance}, która wskazuje na różnicę między liczbą dotychczas rozpatrywanych piłeczek a liczbą dotychczas rozpatrywanych dołków wynosi $0$. Wiemy także, że wszystkie dotychczas rozpatrywane punkty leżą na prawo lub na prostej przechodzącej przez wektor $\overrightarrow{OP}$, gdzie $P$ jest obecnie rozpatrywanym punktem.\\~\\
		Z tego wszystkiego wynika, że wśród punktów znajdujących się na prawo od wektora $\overrightarrow{OP}$ jest dokładnie tyle samo piłeczek, co dołków, a także odcinek $\overline{OP}$ łączy piłeczkę z dołkiem. Wobec tego, znalezione uderzenie $\overline{OP}$ jest uderzeniem zbalansowanym.
	\end{proof}

	\vspace{2em}

	\begin{theorem} \label{theorem:2}
		Algorytm ogólny rozwiązujący zadany problem jest poprawny.
	\end{theorem}
	
	\begin{proof}
		Na mocy twierdzenia \ref{theorem:1} dla dowolnego wejścia możliwe jest znalezienie uderzenia zbalansowanego, po którego lewej (a zarazem prawej) stronie znajduje się tyle samo piłeczek, co dołków. Z dowodu lematu \ref{lemma:1} możemy podzielić te zadanie na dwa podzadania, znajdując dopasowania piłeczek do dołków znajdujących się po lewej i prawej stronie uderzenia zbalansowanego. \\~\\
		Ze względu na to, że $\mathbf{B}_{1} \subsetneq \mathbf{B}$ i $\mathbf{H}_{1} \subsetneq \mathbf{H}$ oraz $\mathbf{B}_{2} \subsetneq \mathbf{B}$ i $\mathbf{H}_{2} \subsetneq \mathbf{H}$, to z każdym zejściem rekursji rozmiar podzadania się zmniejsza. Dla $n = 0$ lub $n = 1$ dochodzimy do dna rekursji i następuje powrót. Na mocy lematu \ref{lemma:1}, ta strategia pozwala na znalezenie prawidłowego planarnego dopasowania piłeczek do dołków na dowolnym poziomie rekursji, w tym na jej samym szczycie.
	\end{proof}
	
	\vspace{2em}
	
	\section{Dowód złożoności algorytmu}
	
	\begin{lemma} \label{lemma:2}
		Algorytm znajdujący uderzenie zbalansowane ma złożoność rzędu co najwyżej $O(n \log n)$.
	\end{lemma}
	
	\begin{proof}
		W celu analizy złożoności, rozważymy poszczególne operacje wykonywane w algorymie. W celu ułatwienia analizy, będziemy się odnosili do poszczególnych etykiet w pseudokodzie:
		\begin{enumerate}
			\item połączenie zbiorów punktów można zrealizować w czasie liniowym (zakładając, że $|\mathbf{B}| = |\mathbf{H}|$ ); zadanie jest tym bardziej proste, że w warunkach wejścia żądamy, aby te punkty były parami różne - $O(n)$;
			\item znalezienie minimalnego punktu, przy założeniu o jednostkowym koszcie porównania, będzie wymagało liniowego przejścia po całej kolekcji punktów - $O(n)$;
			\item translacja każdego punktu o stały wektor ma również złożonosć czasową liniową (przesuwamy każdy punkt z kolekcji) - $O(n)$;
			\item porównanie dwóch punktów $P_{1}$ oraz $P_{2}$ na podstawie kąta między wektorem $OP_{i}$ a osią $OX$ i odległości od $O$ można zrealizować w czasie stałym (np. posługując się monotonicznością funkcji trygonometrycznych na przedziale $\left[0, \pi\right]$); \\ przy tym założeniu, dla metody quicksort z wyborem pivota algorytmem magicznych piątek, posortowanie punktów będzie miało złożoność czasową nadliniową - $O(n \log n)$;
			\item wewnątrz pętli, zostaną wykonane dwa porównania, jedno przypisane i co najwyżej jeden powrót z funkcji; ponieważ pętla w pesymistycznym przypadku przechodzi po całej kolekcji punktów, złożoność tej operacji jest liniowa $O(n)$
		\end{enumerate}
		
		Podsumowując tę krótką analizę widzimy, że najbardziej kosztowną operacją jest posortowanie wszystkich punktów. W związku z tym, złożoność czasowa całego algorytmu znajdowania uderzenie zbalansowanego jest nadliniowa, tzn. wynosi $O(n \log n)$.
	\end{proof}
	
	\vspace{1em}
	
	\begin{lemma} \label{lemma:3}
		W algorytmie ogólnym rozwiązującym problem, funkcja wykonująca ogólny algorytm dla zbiorów o liczności $0$ lub $1$ zostanie wywołana $n$ razy, a dla zbiorów o liczności od $2$ do $n$ — $n - 1$ razy.
	\end{lemma}
	
	\begin{proof}
		Każdym wywołanie funkcji enkapsulującej ogólny algorytm rozwiązania problemu sprowadza się do dwóch możliwych sytuacji:
		\begin{enumerate}
			\item zwrócenia zbioru pustego lub jednoelementowego (warunek stopu rekursji) - wykonanie typu I;
			\item wywoływana jest funkcja znajdująca uderzenie zbalansowane, które później wyznacza podział na mniejsze (zgodnie ze strategią "dziel i zwyciężaj"), dla których rekurencyjnie wywoływana jest ta funkcja - wykonanie typu II;
		\end{enumerate} 
		Możemy zauważyć, że w drzewie rekursji te pierwsze wywołania będą tworzyć liście, zaś te drugie - rodziców w drzewie. Należy pamiętać, że drzewo rekursji jest drzewem 2-regularnym, tzn. każdy węzeł posiada zero lub dwóch potomków.\\~\\
		Ponieważ uderzenie zbalansowane nie trafia do żadnego podzadania, dla którego rekurencyjnie wykonujemy funkcję, każde podzadanie ma rozmiar mniejszy od zadania, z którego powstało, o co najmniej $1$. Ponadto, całkowity rozmiar podzadań jest mniejszy o $1$ niż rozmiar zadania, które zostało podzielone (wynika to wprost ze sposobu podziału). Rekursja zatrzymuje się dla zadań o rozmiarze $0$ lub $1$.\\~\\
		Na podstawie powyższych własności widać, że podziałów w ramach strategii "dziel i zwyciężaj", czyli wykonań funkcji typu II (rodziców w drzewie rekursji) jest $n - 1$, zaś wykonań funkcji typu I (liście w drzewie rekursji) jest o $1$ więcej, czyli $n$.
	\end{proof}	
	
	\vspace{1em}
	
	\begin{theorem} \label{theorem:3}
		Algorytm ogólny rozwiązujący problem ma złożoność rzędu $O(n^{2} \log n)$.
	\end{theorem}
	
	\begin{proof}
		Na podstawie lematu \ref{lemma:3} funkcje ogólne dla wejścia o rozmiarze $0$ lub $1$ zostaną wywołane $n$ razy, zaś pozostałe - $n - 1$ razy. \\~\\
		Funkcje ogólne dla wejścia o rozmiarze $0$ lub $1$ wykonują się w czasie stałym - zwracają zbiór pusty lub wybierają jedyny elementy ze zbiorów i zwracają zbiór jednoelementowy zawierający jedyne możliwe uderzenie (etykieta 1).\\~\\
		Dla wywołań funkcji ogólnej dla wejścia o rozmiarze większym niż $1$ najbardziej kosztowną operacją jest znajdowanie uderzenia zbalansowanego (etykieta 2) - zajmuje $O(n \log n)$ czasu. Sprawdzenie, czy punkt jest położony zgodnie lub przeciwnie do ruchu wskazówek zegara względem jakiegoś wektora można wykonać za pomocą prostego iloczynu wektorowego w czasie stałym. Podział pozostałych punktów na stronę zgodną i przeciwną do ruchu wskazówek można więc wykonać w czasie liniowym (etykieta 3). Podobnie, scalenie rozłącznych zbiorów w jeden (etykieta 4) również wymaga liniowego nakładu czasu. To wszystko prowadzi do wniosku, że wywołanie typu II funkcji ogólnej ma złożoność czasową $O(n \log n)$).
		\\~\\
		Wobec tego, cały algorytm wykonuje się w czasie $(n - 1) \cdot O(n \log n) + n \cdot O(1)$. Jest to równoważne temu, że algorytm posiada złożoność algorytmiczną czasową $O(n^{2} \log n)$.
	\end{proof}

	\pagebreak
	
	\section{Dokumentacja techniczna}
	
	\subsection{Architektura rozwiązania}
	
	W celu reprezentowania zadania o określonym rozmiarze została stworzona klasa \textit{Task}. Zawiera ona kolekcje obiektów typu \textit{Ball} i \textit{Hole}. Obiekty te dziedziczą po abstrakcyjnej klasie \textit{Point} i odpowiadają piłkom oraz dołkom z rozważanego problemu. Pochodne klasy {Point} zawierają pole informujące o typie punktu oraz wartość pola \textit{Value} odpowiednią dla tego typu.
	
	\begin{figure}[H]
		\centering	
 			\includegraphics[scale=0.5]{images/structure_diagram.png}
 			\caption{Struktura obiektów.}
 		\label{fig:structure_diagram}
	\end{figure}	

	Dla realizacji algorytmu zostały zrealizowane interfejsy \textit{IBalancedHitFinder} oraz \textit{IPlanarMatchingFinder}. Ten pierwszy implementowany przez klase \textit{BalancedHitFinder} i posiada metodę \textit{FindBalancedHit} służąca do znalezienia zbalansowanego uderzenia, przyjmując jako argumenty kolekcje obiektów odpowiadające piłkom oraz dołkom. \\~\\
	Z interfejsu tego korzysta klasa \textit{PlanarMatchingFinder}, która w konstruktorze przyjmuje obiekt implementujący go. Za pomocą metody \textit{FindPlanarMatching} wywoływane jest rozwiązanie problemu i w przypadku powodzenia wynik jest zwracany za pomocą klasy \textit{Matching}, która zawiera kolekcję obiektów \textit{Hit} zawierających informacje o parze punktów (piłka i dołek) pomiędzy którymi przebiega uderzenie.

	\begin{figure}[H]
		\centering
 			\includegraphics[scale=0.6]{images/algorithm_diagram.png}
 			\caption{Obiekty realizujące algorytm.}
 		\label{fig:algorithm_diagram}
	\end{figure}	

	Operacje wejścia wykonywane są za pomocą klasy \textit{TextFileTaskParser} implementującej interfejs \textit{ITaskParser}. Metoda \textit{Parse} przyjmuje jako argument ściezkę pliku tekstowego w którym zapisane są w odpowiednim formacie dane wejściowe dla zadania i zwraca obiekt \textit{Task} reprezentujący to zadanie. \\~\\
	Zapisywanie obecnego zadania wykonywane jest za pomocą klasy \textit{TextFileTaskSaver} implementującej interfejs \textit{TextFileTaskSaver}. Za pomocą metody \textit{Save}, któRa przyjmuje jako argument obiekt typu \textit{Task} zawierający dane zadania i zapisuje je w określonym formacie do pliku tekstowego.\\
	Natomiast zapisywanie znalezionego rozwiązania realizuje klasa \textit{TextFileMatchingPrinter} implementująca interfejs \textit{IMatchingPrinter}. Zawiera on metodę \textit{Print}, która jako argument przyjmuje obiekt klasy \textit{Matching} i zapisuje w określonym formacie zawarte w nim uderzenia do pliku tekstowego.
		
	\subsection{Testy jednostkowe}
	
	W projekcie \textit{GolfAppTests} znajdują się test jednostkowe dotyczące projektu \textit{GolfApp}. Sprawdzają one poprawność logiki klas wykonujących algorytm oraz zachowanie w przypadku różnych (także błędnych) argumentów metod publicznych. \\
	Test zostały wykonane przy wykorzystaniu frameworku do tesowanie \textit{xUnit}. Poniżej został załączony zrzut ekranu przedstwiające listę testów i status ich wykonania. 
	
	\begin{figure}[H]
		\centering
			\includegraphics[scale=0.6]{images/unittests.png}
 			\caption{Testy jednostkowe.}
 		\label{fig:unit_tests}
	\end{figure}	
	
	\subsection{Testy akceptacyjne i wydajnościowe}
	
	\pagebreak
	
	\section{Instrukcja użytkownika}
	
	W celu uruchomienia aplikacji należy uruchomić plik \textit{GolfUI.exe}. Po poprawnym uruchomieniu powinno pojawić się okno początkowe programu. 
	
	\begin{figure}[H]
		\centering	
 			\includegraphics[scale=0.6]{images/initial.png}
 			\caption{Okno początkowe.}
 		\label{fig:algorithm_diagram}
	\end{figure}	
	
	W menu dostępne są elementy:
	
	\begin{itemize}
		\item{Task}
		
			\begin{figure}[H]
				\centering	
 				\includegraphics[scale=0.6]{images/task.png}
 				\caption{Opcje menu Task.}
 				\label{fig:task_menu}
			\end{figure} 
			
			\begin{itemize}
				\item Opcja \textit{Load} umożliwia wskazanie pliku w którym znajdują się dane wejściowe zadania i załadowanie go do programu. Dane te powinny być w następującym formacie:
	
				\begin{verbatim}
					[N - liczba piłek]
					[Współrzędna X 1-szej piłki], [Współrzędna Y 1-szej piłki]
					...
					[Współrzędna X N-tej piłki], [Współrzędna Y N-tej piłki]
					[Współrzędna X 1-szego dołka], [Współrzędna Y 1-szego dołka]
					...
					[Współrzędna X N-tego dołka], [Współrzędna Y N-tego dołka]
				\end{verbatim}	
			
				\textbf{Przykład:} 		
				\begin{verbatim}
					3
					10,10
					0, 5
					20,50
					50,20
					90,30
					5, 5
				\end{verbatim}
		
				\item Opcja \textit{Save} umożliwia zapisanie obecnego zadania do wskazanego pliku. Zapis następuje w formacie wspomnianym przy poprzedniej opcji.
			\end{itemize}
		
		\item{Matching}
			\begin{figure}[H]
				\centering	
		 			\includegraphics[scale=0.6]{images/matching.png}
		 			\caption{Opcje menu Matching.}
		 		\label{fig:menu_matching}
			\end{figure}	
			
			\begin{itemize}
				\item Opcja \textit{Find} znajduje uderzenia pomiędzy dołkami i piłkami zdefiniowanymi w obecnym zadaniu.
				\item Opcja \textit{Save} zapisuje rozwiązanie problemu (uderzenia), jeśli takie zostało już znalezione, do wskazanego pliku tekstowego.
				\item Opcja \textit{Clear} czyści obecnie znalezione rozwiązanie.
			\end{itemize}
	\end{itemize}
	
	Punkty z zadania prezentowane są za pomocą listy znajdującej się w prawej części ekranu (\textit{List of points}). Na lisćie wyświetlany jest jednak tylko jeden typ punktow, który można wybrać za pomocą rozwijanej listy \textit{Points type}.
		
	Za pomocą przycisku \textit{Add point} możliwe jest dodanie nowych punktów do obecnie wybranej listy. Należy podać współrzędne nowego punkt w polach tekstowych znajdujących się obok przycisku i dodać punkt klikając przycisk. W przypadku, gdy punkt o podanych współrzędnych istnieje na którejś z list, zostanie wyświetlony komunikat o błędzie. 
	
	\begin{figure}[H]
		\centering	
		\includegraphics[scale=0.3]{images/add.png}
 		\caption{Dodawanie punktu.}
 		\label{fig:algorithm_diagram}
	\end{figure}	

	Usuwanie punktu można wykonać poprzez kliknięcie na któryś z elementów listy prawym przyciskiem myszy. Zostanie wyświetlone menu kontekstowe z dostępną opcją \textit{Remove}. Jeżeli ta opcja zostanie wybrana, wskazany element zostanie usunięty z listy punktów.
		
	\begin{figure}[H]
		\centering
 		\includegraphics[scale=0.3]{images/remove.png}
 		\caption{Usuwanie punktu.}
 		\label{fig:algorithm_diagram}
	\end{figure}

	\subsection{Podział pracy}
	
	W dokumentacji projektowej, Mateusz Rymuszka wykonał opis algorytmu, dowiódł poprawności i złożoności obliczeniowej algorytmu oraz przygotował instrukcję użytkownika. Piotr Piwowarski opisał algorytm wraz z rozmiarem zadania, wykonał pseudokod algorytmu, opisał sposób prezentacji danych wejściowych i wyjściowych, rozwiązał przykładowe zadanie oraz wykonał dokuementację techniczną projektu.\\~\\
	W projekcie programistycznym, Mateusz Rymuszka zaimplementował algorytm, zaprojektował struktury używane do obliczania programu, stworzył prototyp interfejsu graficznego oraz przygotował API do testów. Piotr Piwowarski zaimplementował obsługę wejścia i wyjścia w programie, znacząco udoskonalił interfejs graficzny oraz przygotował testy jednostkowe, wydajnościowe i akceptacyjne.

	\section{Wnioski}
	
	

\end{document}