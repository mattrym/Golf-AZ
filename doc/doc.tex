\documentclass[10pt,a4paper]{article}
\usepackage[T1]{fontenc}
\usepackage[utf8]{inputenc}
\usepackage{amssymb}
\usepackage{graphicx}
\usepackage[polish]{babel}
\usepackage{hyperref}
\usepackage[paper=a4paper,margin=1in]{geometry}
\usepackage{tabularx} 
\usepackage{helvet}
\usepackage{amsfonts}
\usepackage{amsthm}
\usepackage{mathtools}
\usepackage{algpseudocode}
\usepackage{tikz}
\usepackage{tkz-graph}
\usepackage{listings,xcolor}
\renewcommand{\familydefault}{\sfdefault}
\setlength{\parindent}{0pt}
\newtheorem{theorem}{Twierdzenie}
\newtheorem{lemma}{Lemat}
\newtheorem{invariant}{Niezmiennik}
\newtheorem{conculsion}{Wniosek}

\begin{document}
	\begin{titlepage}
		\newgeometry{top=1in,bottom=1in,right=1.5in,left=1.5in}
		\begin{center}
			{\fontsize{14}{12}\selectfont Politechnika Warszawska \\ Wydział Matematyki i Nauk Informacyjnych}
			
		\end{center}
		
		\vspace{1cm}
		\begin{center}
			\includegraphics[width=0.3\textwidth]{images/logo.png}
		\end{center}
		\vspace{3cm}
		
		\begin{center}
			\textbf{{\fontsize{26}{12}\selectfont Algorytmy Zaawansowane}}
			
			\vspace{2cm}
			\textbf{{\fontsize{22}{12}\selectfont Dokumentacja wstępna}}
			\vspace{1cm}
			
			\textbf{{\fontsize{13.5}{12}\selectfont Piotr Piwowarski, Mateusz Rymuszka}}
			
			\vspace{6cm}
			\textbf{{\fontsize{13.5}{12}\selectfont \today}}
		\end{center}  
	\end{titlepage}
	
	{\fontsize{13.5}{12}\selectfont
		\tableofcontents
		\vspace{1cm}
		{\renewcommand{\arraystretch}{2.0}
			
			\begin{tabularx}{\textwidth}{|l|l|X|l|}
				\hline 
				Data & Autor & Opis zmian & Wersja\\
				\hline
				10.03.2018 & Mateusz Rymuszka & Pierwsza wersja dokumentu & 1.0 \\
				\hline
				11.03.2018 & Mateusz Rymuszka & Opis problemu, zadania, algorytmu oraz wejścia/wyjścia & 1.1 \\
				\hline
				15.03.2018 & Mateusz Rymuszka & Opis problemu, zadania, algorytmu oraz wejścia/wyjścia & 1.2 \\
				\hline
				18.03.2018 & Mateusz Rymuszka & Dowód poprawności algorytmu algorytmu & 1.3 \\
				\hline
				19.03.2018 & Mateusz Rymuszka & Dowód złożoności działania algorytmu & 1.4 \\
				\hline
			\end{tabularx} 
	}}
	
	\newpage
	
	\section{Słownik pojęć}
	
	W niniejszej dokumentacji wykorzystane są poniższe pojęcia:
	\begin{itemize}
		\item \textbf{piłeczka} - punkt na płaszczyźnie dwuwymiarowej ze zbioru $\mathbb{B}$;
		\item \textbf{dołek} - punkt na płaszczyźnie dwuwymiarowej ze zbioru $\mathbb{H}$;
		\item \textbf{uderzenie} - odcinek łączący piłeczkę i dołek
		\item \textbf{uderzenie zbalansowane} - uderzenie, w którym po każdej ze stron półpłaszczyzn rozgraniczonych przez prostą zawierającą te uderzenie znajduje się tyle samo piłeczek, co dołków
		\item \textbf{dopasowanie} - zbiór uderzeń
		\item \textbf{planarne dopasowanie} - dopasowanie, w którym żadna para uderzeń nie posiada punktu wspólnego
	\end{itemize}
	
	\section{Problem}
	
	\textbf{Temat projektu: GOLF} \\~\\
	Na polu golfowym znajduje się $n$ piłeczek oraz $n$ dołków na piłeczki. Golfiści chcą jednocześnie umieścić każdy swoją piłeczkę w którymś z dołków. W tym celu ustalają między sobą, który celuje do którego dołka, ale w taki sposób, by tory ich	piłeczek się nie przecinały, co gwarantuje brak zderzeń piłeczek.\\~\\ 
	Załóżmy, że piłeczki i dołki są punktami na płaszczyźnie oraz żadne trzy z tych punktów nie są współliniowe, a tory piłeczek są odcinkami prostej. Przedstaw działający w czasie $O(n^2\log n)$ algorytm przydzielania piłeczek do dołków tak, aby żadne dwa tory piłeczek się nie przecinały.\\~\\
	\textit{Wskazówka:} W czasie $O(n\log n)$ można znaleźć prostą przechodzącą przez jedną piłeczkę i jeden dołek taką, że liczby piłeczek i dołków po tej samej stronie tej prostej są równe.
	
	\section{Zadanie oraz jego rozmiar}
	
	\subsection{Zadanie}
	Dwa zbiory $B$ oraz $H$ różnych, trójkami niewspółliniowych punktów o liczności $n$ każdy z przestrzeni $\mathbb{R}^{2}$ (reprezentujących odpowiednio położenia piłeczek oraz położenia dołków).
	
	\subsection{Rozmiar zadania}
	
	Całkowita liczba punktów wynosi $2n$. Biorąc jednak pod uwagę fakt, że rozpatrujemy pary piłka - dołek oraz prostotę dalszej analizy możemy przyjąć. że rozmiar zadania nie zależy od stałej i wynosi $n$.
	
	\section{Sposób rozwiązania}
	
	\subsection{Opis}
	
	Algorytm bazuje na strategii dziel i zwyciężaj. Dla zadanego zbioru wyszukujemy uderzenie $S$, po którego lewej lub prawej stronie (nie ma to znaczenia ze względu na równoliczność zbiorów $\mathbb{B}$ oraz $\mathbb{H}$) znajduje się dokładnie tyle samo piłeczek, co dołków. Następnie wywołujemy ten algorytm osobno dla piłeczek i dołków znajdujących się po lewej stronie, a osobno po prawej. \\~\\
	Wywołując algorytm dla pustych zbiorów piłeczek i dołków, otrzymamy zbiór pusty. Wywołując algorytm dla jednej piłeczki i jednego dołka, zwracamy uderzenie składające się z tej piłeczki i tego dołka. Wszystkie znalezione przez ten algorytm uderzenia tworzą rozwiązanie zadania. \\~\\
	Aby znaleźć uderzenie $S$, po którego jednej ze stron znajduje się dokładnie tyle samo piłeczek, co dołków, zastosujemy algorytm podobny w budowie do algorytmu Grahama znajdowania otoczki wypukłej punktów w przestrzeni dwuwymiarowej. Wyszukujemy punkt o najmniejszej współrzędnej rzędnej (jeżeli jest ich kilka, to wybieramy ten o najmniejszej współrzędnej odciętej) $O$ i przesuwamy początek układu współrzędnych do tego punktu. Następnie sortujemy pozostałe punkty po kącie między wektorem od początku układu do tego punktu a osią $OX$ (ponieważ dowolne trzy punkty z obu zbiorów nie są współliniowe, to każdy punkt będzie miał inną wartość tego kąta). \\~\\
	Każdemu punktowi, w zależności od tego, czy jest piłeczką czy dołkiem, przyporządkowana jest wartość liczbowa: dla piłeczki jest to $1$, zaś dla dołka $-1$. Inicjalizujemy zmienną pomocniczą wartością zerową i następnie przechodzimy po posortowanej liście punktów, dodając do tej zmiennej wartość aktualnego punktu. W momencie, gdy zmienna pomocnicza osiągnie ponownie wartość $0$, algorytm kończy działanie, a ostatnio rozpatrywany punkt tworzy z punktem $O$ uderzenie, które spełnia wspomniany warunek.\\~\\
	Po znalezieniu uderzenia zbalansowanego oraz znalezieniu rozwiązań dla zmniejszych zadań na lewo i na prawo od tego uderzenia, jako wynik zwracamy sumę teoriomnogościową rozwiązania dla punktów po lewej, zbiór jednoelementowy zawierający znalezione uderzenie zbalansowane oraz rozwiązanie dla punktów po prawej.
	
	\subsection{Pseudokod}

	\subsubsection{Pseudokod ogólnego algorytmu}

	\begin{algorithmic}
		\State \textbf{Wejście:} zbiór piłeczek $\mathbf{B}$ oraz dołków $\mathbf{H}$  \\
		
		\Function {FindPlanarMatching}{$\mathbf{B}$, $\mathbf{H}$}
			\If{$|\mathbf{B}| = |\mathbf{H}| = 0$} \Comment{(1)}
				\State \Return $\emptyset$ 
			\EndIf
			\\
			\If{$|\mathbf{B}| = |\mathbf{H}| = 1$} \Comment{(1)}
				\State $B \coloneqq $ punkt z $\mathbf{B}$
				\State $H \coloneqq $ punkt z $\mathbf{H}$
				\State \Return $\left\{\overline{BH}\right\}$ 
			\EndIf
			\\
			\State $\overline{LS} \coloneqq$ \Call{FindBalancedMatch}{$\mathbf{B}$, $\mathbf{H}$} \Comment{(1)}
			\\
			\State $\mathbf{B}_{1} \coloneqq $ punkty z $\mathbf{B}$ na lewo od $LS$ \Comment{(2)}
			\State $\mathbf{H}_{1} \coloneqq $ punkty z $\mathbf{H}$ na lewo od $LS$ \Comment{(2)}
			\State $\mathbf{B}_{2} \coloneqq $ punkty z $\mathbf{B}$ na prawo od $LS$ \Comment{(2)}
			\State $\mathbf{H}_{2} \coloneqq $ punkty z $\mathbf{H}$ na prawo od $LS$ \Comment{(2)}
			\\
			\State $\mathbf{M}_{1} \coloneqq $ \Call{FindPlanarMatching}{$\mathbf{B}_{1}$, $\mathbf{H}_{1}$} \Comment{(3)}
			\State $\mathbf{M}_{2} \coloneqq $ \Call{FindPlanarMatching}{$\mathbf{B}_{2}$, $\mathbf{H}_{2}$} \Comment{(3)}
			\State \Return $\mathbf{M}_{1} \cup \left\{\overline{LS}\right\} \cup \mathbf{M}_{2}$
		\EndFunction
	\end{algorithmic}

	\subsubsection{Pseudokod algorytmu znajdującego uderzenie zbalansowane}
	
	\begin{algorithmic}
		\State \textbf{Wejście:} zbiór piłeczek $\mathbf{B}$ oraz dołków $\mathbf{H}$  \\
		
		\Function {FindBalancedMatch}{$\mathbf{B}$, $\mathbf{H}$}
			\State $\mathbf{P} \coloneqq \mathbf{B} \cup \mathbf{H}$ \Comment{(2)}
			\State $O \coloneqq$ początek układu współrzędnych
			\State $L \coloneqq$ najbardziej położony na lewo spośród najniższych punktów z $\mathbf{P}$ \Comment{(3)}
			\\
			\State przesuń wszystkie punkty z $\mathbf{P}$ o wektor $\overrightarrow{LO}$ \Comment{(4)}
			\State posortuj $\mathbf{P}$ leksykograficznie po kącie między wektorem $\overrightarrow{OP}$ a osią $OX$ i odległości od $O$ \Comment{(6)} 
			\\
			\State balance $\coloneqq 0$
			\ForAll{$P \in \mathbf{P}$} \Comment{(5)} 
				\If{$P \in \mathbf{B}$}
					\State balance $\mathrel{{+}{=}} 1$
				\Else
					\State balance $\mathrel{{-}{=}} 1$
				\EndIf
				\If{balance $= 0$}
					\State \Return $\overline{LP}$
				\EndIf
			\EndFor
		\EndFunction
	\end{algorithmic}
	
	\section{Rozwiązanie przykładowego zadania}
	
	\section{Format danych wejściowych oraz format wyjścia}
	
	\subsection{Wejście}
	
	Plik wejściowy posiada następujące informacje:
	\begin{itemize}
		\item liczbę piłeczek lub liczbę dołków, czyli $n$ w pierwszej linii;
		\item współrzędne piłeczek rozdzielone przecinkami, każda piłeczka jest opisana w nowej linii;
		\item współrzędne dołków rozdzielone przecinkami, każdy dołek jest opisany w nowej linii;
	\end{itemize}

	\textbf{Przykład:} 
	
	\subsection{Wyjście}
	
	Plik wyjściowy posiada następujące informacje:
	\begin{itemize}
		\item indeks piłeczki w pliku wejściowym oraz indeks dołka w pliku wejściowym, które tworzą uderzenie, rozdzielone spacją; każde uderzenie jest opisane w nowej linii;
	\end{itemize}
	
	\textbf{Przykład:} 
	
	\section{Dowód poprawności rozwiązania}
	
	\begin{lemma}\label{lemma:1}
		Niech $\mathbf{B}$ będzie zbiorem piłeczek, zaś $\mathbf{H}$ zbiorem dołków oraz niech $|\mathbf{B}| = |\mathbf{H}|$ . Jeżeli $LS$ jest uderzeniem takim, że po jego lewej (prawej) stronie znajduje się tyle samo piłeczek co dołków, to dopasowanie będącego sumą teoriomnogościową:
		\begin{itemize}
			\item planarnego dopasowania $\mathbf{M}_{1}$ piłeczek na lewo od $LS$ z dołkami na lewo od $LS$
			\item zbioru jednoelementowego $\left\{LS\right\}$
			\item planarnego dopasowania $\mathbf{M}_{2}$ piłeczek na prawo od $LS$ z dołkami na prawo od $LS$
		\end{itemize}
		jest planarne.
	\end{lemma}

	\begin{proof}
		Dla przypadku dla $n = 1$  teza jest oczywista. \\~\\
		Zauważmy, że jeżeli $LS$ spełnia założenie lematu, to z równoliczności $\mathbf{B}$ i $\mathbf{H}$ oraz z tego założenia wynika, że jeżeli po jednej stronie uderzenia jest tyle samo piłeczek, co dołków, to po drugiej stronie również jest tyle samo piłeczek, co dołków (gdyż, w sumie, również musi być tyle samo piłeczek, co dołków). \\~\\
		Niech $\mathbf{B}_{1}$ i $\mathbf{H}_{1}$ będą zbiorami odpowiednio piłeczek i dołków na lewo od $LS$, zaś $\mathbf{B}_{2}$ i $\mathbf{H}_{2}$ będą zbiorami piłeczek i dołków na lewo od $LS$. Ponieważ $|\mathbf{B}_{1}| = |\mathbf{H}_{1}|$ i $|\mathbf{B}_{2}| = |\mathbf{H}_{2}|$, to pary $\mathbf{B}_{1}$ i $\mathbf{H}_{1}$ oraz $\mathbf{B}_{2}$ i $\mathbf{H}_{2}$ również są zadaniami należącymi do zbioru zadań rozpatrywanego problemu. \\~\\
		Zakładając, że dopasowanie $\mathbf{M}_{1}$ jest planarne wiemy, że dla dowolnej pary uderzeń z $\mathbf{M}_{1}$ te uderzenia nie mają punktu wspólnego. Dodatkowo, skoro $\mathbf{M}_{1}$ to zbiór uderzeń piłeczek znajdujących się na lewo od $LS$ do dołków znajdujących się na lewo od $LS$, to dowolne uderzenie $\mathbf{M}_{1}$ nie ma punktu wspólnego z $LS$. Analogicznie, dowolna para uderzeń z $\mathbf{M}_{2}$ nie ma punktu wspólnego oraz żadne uderzenie z $\mathbf{M}_{2}$ nie ma punktu wspólnego z $LS$. Wobec tego, dowolna para uderzeń ze zbioru $\mathbf{M}_{1} \cup \left\{LS\right\} \cup \mathbf{M}_{2}$ nie ma punktu wspólnego. Ponadto, uderzenia z tego zbioru pokrywają zbiór punktów $\mathbf{B} \cup \mathbf{H}$. Z tego wprost wynika, że dopasowanie $\mathbf{M}_{1} \cup \left\{LS\right\} \cup \mathbf{M}_{2}$	jest planarne.
	\end{proof}

	\begin{theorem} \label{theorem:1}
		Algorytm znajdujący uderzenie zbalansowane znajduje takie dopasowanie dla dowolnych skończonych i równolicznych zbiorów $\mathbf{B}$ piłeczek i $\mathbf{H}$ dołków.
	\end{theorem}

	\begin{proof}
		Dla $n = 0$ oraz $n = 1$ dowód jest trywialny i wynika wprost z definicji uderzenia zbalansowanego, więc go pomijamy. \\~\\
		Operacje początkowe mają na celu przygotować punkty do wykonania właściwej części algorytmu. Wynikające z nich własności pomogą nam w udowodnieniu niezmienników głównej pętli. Należą do nich:
		\begin{enumerate}
			\item znaleźliśmy najniższy, najbardziej na lewo punkt $L$ spośród wszystkich piłeczek i dołków, 
			\item przesunęliśmy do punktu $L$ początek układu współrzędnych $O$
			\item posortowaliśmy punkty z $\mathbf{P}$ po kącie między wektorem $\overrightarrow{OP}$ a osią $OX$ oraz odległości od $O$
		\end{enumerate}
		Ostatni punkt nie wymaga sortowania wszystkich punktów po odległości od $O$, gdyż z założeń żadne trzy punkty z nie są współliniowe. Wymusza to jedynie, aby punkt $O = L$ był pierwszy w kolejności. \\~\\
		W celu dowiedzenia poprawności naszego algorytmu, pokażemy, że zachodzą pewne niezmienniki pętli głównej naszego algorytmu.

		\begin{invariant} \label{invariant:1}
			Algorytm nie zakończy się w pierwszym kroku pętli.
		\end{invariant}

		\begin{proof}
			Dość oczywisty - w tym kroku zmienna balance będzie wynosić $-1$ lub $1$, więc algorytm nie przerwie pętli i będzie się wykonywał dalej. 
		\end{proof}

		\begin{invariant} \label{invariant:2}
			Algorytm zakończy się w którymś z kroków pętli.
		\end{invariant}

		\begin{proof}
			Zmienna balance jest zainicjowana zerem przed wejściem do pętli. Następnie, w każdym kroku pętli jest ona zwiększana o jeden (gdy punkt rozpatrywany w iteracji jest piłeczką) lub zmniejszana od jeden (gdy punkt rozpatrywany w iteracji jest dołkiem). \\~\\
			Gdyby algorytm nie zakończył się w którejś z pętli, oznaczałoby to, że zmienna balance nie przyjęła wartości zero po żadnym z kroków pętli, w szczególności - ostatnim. Ale ponieważ  = , to po ostatnim kroku pętli rozpatrzyliśmy wszystkie piłeczki i wszystkie dołki, których jest tyle samo. Oznacza to, że do zmiennej balance dodaliśmy dokładnie tyle, co od niej odjęliśmy, czyli wynosi tyle, ile jej wartość początkowa, czyli $0$. Sprzeczność - pętla wykona najpóźniej w ostatnim kroku pętli. 
		\end{proof}

		\begin{invariant} \label{invariant:3}
			W $k$-tym kroku pętli (gdzie $k \geq 3)$ na prawo (zgodnie z ruchem wskazówek zegara) od odcinka $\overline{OP}$ znajdują się punkty rozpatrywane w krokach od drugiego do $k-1$-tego.
		\end{invariant}

		\begin{proof}
			Punkty są posortowane rosnąco po kącie między wektorem $\overrightarrow{OP}$ a osią $OX$.  Wobec tego, dla różnych punktów $P_{i}$ oraz $P_{j}$, dla których te kąty wynoszą odpowiednio $\alpha_{i}$ oraz $\alpha_{j}$, jeżeli $i < j$, to $\alpha_{i} < \alpha_{j}$. \\~\\
			Wobec tego, kąt między wektorem $\overrightarrow{OP_{ji}}$ oraz $\overrightarrow{OP_{j}}$ wynosi $\alpha_{i} - \alpha_{j} < 0$, zatem punkt $P_{j}$ jest położony zgodnie z ruchem wskazówek zegara ("na prawo") w stosunku do punktu $P_{i}$, gdzie $O$ jest początkiem układu współrzędnych.
		\end{proof}

		\begin{invariant} \label{invariant:4}
			Zmienna balance po wykonaniu $k$-tego kroku pętli (poza pierwszym krokiem pętli) dla punktu $P$ wskazuje na różnicę między liczbą piłeczek a liczbą dołków na półpłaszczyźnie na prawo od prostej zawierającej odcinek $\overline{OP}$, włącznie z tą prostą.
		\end{invariant}

		\begin{proof}[Dowód (indukcyjny)]
			Po pierwszym kroku pętli zmienna balance wynosi $-1$, gdy najniższy, najbardziej na lewo punkt jest dołkiem, albo $1$, gdy ten punkt jest piłeczką. 
			\begin{enumerate}
				\item Dla $k = 2$ nie ma punktów na prawo od wektora $OP$. Zmienna balance jest wówczas sumą wartości dla pierwszego i drugiego punktu. Może wynieść:
				\begin{enumerate}
					\item $2$, gdy oba punkty są piłeczkami,
					\item $0$, gdy jeden z punktów jest piłeczką, a drugi dołkiem,
					\item $-2$, gdy oba punkty są dołkami.
				\end{enumerate}
				\item Założmy, że dla pewnego $k < 2n$ spełniona jest teza niezmiennika. Wówczas rozpatrzmy krok $(k+1)$-ty, w którym rozpatrujemy punkt $P$. \\~\\
				Na mocy niezmiennika \ref{invariant:3} na prawo od wektora $\overrightarrow{OP}$ znajdują się punkty rozpatrywane w krokach od drugiego do $k$-tego. Z kolei na mocy założenia indukcyjnego, zmienna balance przed wykonaniem $(k+1)$-go kroku wskazuje na różnicę między liczbą piłeczek a liczbą dołków dla punktów rozpatrywanych w poprzednich krokach algorytmu. \\~\\
				Zmienna balance zwiększy się o $1$, gdy rozpatrywany punkt $P$ będzie piłeczką, lub zmniejszy się o $1$, gdy będzie on dołkiem. Tak samo, różnica między liczbą piłeczek a dołków również zwiększy się o $1$, gdy rozpatrywany punkt $P$ będzie piłeczką, lub zmniejszy się o $1$, gdy będzie on dołkiem. Wobec tego, po wykonaniu $(k+1)$-tego kroku zmienna balance nadal będzie wskazywać na różnicę między liczbą piłeczek a liczbą dołków dla punktów rozpatrywanych w krokach od pierwszego do $(k+1)$-tego. 
			\end{enumerate}
			
		\end{proof}
	
		\begin{invariant}
			Algorytm znajduje odcinek będący uderzeniem.
		\end{invariant}
	
		\begin{proof}
			Założmy, że znaleziony odcinek, nie jest uderzeniem. Wobec tego, w pierwszym i ostatnim kroku algorytmu wartość balance albo się zwiększyła, albo się zmniejszyła. Dla ustalenia uwagi przyjmijmy, że w tych krokach rozpatrywano piłeczki i wartości zmiennej balance była zwiększana.\\~\\
			Po pierwszym kroku wartość zmiennej balance wynosiła $1$. Z kolei, jeżeli po ostatnim kroku zmienna balance wynosiła $0$, to przed nim musiała wynosić $-1$ (w tym kroku rozpatrywaliśmy piłeczkę). W związku z tym, w krokach od drugiego do przedostatniego wartość zmiennej balance zmieniała się od $1$ do $-1$. Ponieważ w jednym kroku zmienna balance może się mniejszyć maksymalnie o $1$, to po pewnym kroku między drugim a przedostatnim wartość zmiennej balance wynosiła $0$. Sprzeczność - algorytm powinien zakończyć się w momencie, gdy zmienna balance po raz pierwszy osiągnie wartość zerową.
		\end{proof}

		Na podstawie tych niezmienników wiemy, że algorytm zakończy się w którymś z kroków (poza pierwszym). Wiemy także, że po tym kroku zmienna balance, która wskazuje na różnicę między liczbą dotychczas rozpatrywanych piłeczek a liczbą dotychczas rozpatrywanych dołków wynosi $0$. Wiemy także, że wszystkie dotychczas rozpatrywane punkty leżą na prawo lub na prostej przechodzącej przez wektor $\overrightarrow{OP}$, gdzie $P$ jest obecnie rozpatrywanym punktem. Z tego wynika, że wśród punktów znajdujących się na prawo od wektora $\overrightarrow{OP}$ jest dokładnie tyle samo piłeczek, co dołków, a także odcinek $OP$ łączy piłeczkę z dołkiem. Wobec tego, znalezione uderzenie $OP$ jest uderzeniem zbalansowanym.

	\end{proof}

	\begin{theorem}
		Algorytm ogólny rozwiązujący zadany problem jest poprawny.
	\end{theorem}
	
	\begin{proof}
		Na mocy twierdzenia \ref{theorem:1} dla dowolnego wejścia możliwe jest znalezienie uderzenia zbalansowanego, po którego lewej (a zarazem prawej) stronie znajduje się tyle samo piłeczek, co dołków. Z dowodu lematu \ref{lemma:1} możemy podzielić te zadanie na dwa podzadania, znajdując dopasowania piłeczek do dołków znajdujących się po lewej i prawej stronie uderzenia zbalansowanego. \\~\\
		Ze względu na to, że $\mathbf{B}_{1} \subsetneq B$ i $\mathbf{H}_{1} \subsetneq H$ oraz $\mathbf{B}_{2} \subsetneq B$ i $\mathbf{H}_{2} \subsetneq H$, to z każdym zejściem rekursji rozmiar podzadania się zmniejsza. Dla $n = 1$ dochodzimy do dna rekursji i następuje powrót. Na mocy lematu \ref{lemma:1}, ta strategia pozwala na znalezenie prawidłowego planarnego dopasowania piłeczek do dołków na dowolnym poziomie rekursji, w tym na jej samym szczycie. trzy punkty nie są współliniowe.
	\end{proof}
	
	\section{Złożoność}
	
	\begin{lemma}
		Algorytm znajdujący uderzenie zbalansowane ma złożoność rzędu co najwyżej $O(n \log n)$.
	\end{lemma}
	
	\begin{proof}
		W celu analizy złożoności, rozważymy poszczególne operacje wykonywane w algorymie. W celu ułatwienia analizy, będziemy się odnosili do poszczególnych etykiet w pseudokodzie:
		\begin{enumerate}
			\item obsługa przypadków wyjątkowych jest trywialna i wykonuje się w czasie stałym - $O(1)$;
			\item połączenie zbiorów punktów można zrealizować w czasie liniowym (zakładając, że $|\mathbf{B}| = |\mathbf{H}|$ ); dodatkowo, zadanie jest o tyle łatwe, że w warunkach wejścia uwzględniamy fakt, aby te punkty były parami różne - $O(n)$;
			\item znalezienie minimalnego punktu, przy założeniu o jednostkowym koszcie porównania, będzie wymagało liniowego przejścia po całej kolekcji punktów - $O(n)$;
			\item translacja każdego punktu o stały wektor ma również złożonosć czasową liniową (przesuwamy każdy punkt z kolekcji) - $O(n)$;
			\item porównanie dwóch punktów $P_{1}$ oraz $P_{2}$ na podstawie kąta między wektorem $OP_{i}$ a osią $OX$ i odległości od $O$ można zrealizować w czasie stałym (np. posługując się monotonicznością funkcji trygonometrycznych na przedziale $\left[0, \pi\right]$); \\ przy tym założeniu, dla metody quicksort z wyborem pivota algorytmem magicznych piątek, posortowanie punktów będzie miało złożoność czasową nadliniową - $O(n \log n)$;
			\item wewnątrz pętli, zostaną wykonane dwa porównania, jedno przypisane i co najwyżej jeden powrót z funkcji; ponieważ pętla w pesymistycznym przypadku przechodzi po całej kolekcji punktów, złożoność tej operacji jest liniowa $O(n)$
		\end{enumerate}
		
		Podsumowując tę krótką analizę widzimy, że najbardziej kosztowną operacją jest posortowanie wszystkich punktów. W związku z tym, złożoność czasowa całego algorytmu znajdowania uderzenie zbalansowanego jest nadliniowa, tzn. wynosi $O(n \log n)$.
	\end{proof}
	
	\begin{lemma}
		W algorytmie ogólnym rozwiązującym problem, wysokość drzewa rekursji wynosi wynosi $n$.
	\end{lemma}
	
	\begin{proof}
		W każdym wywołaniu funkcji enkapsulującej ogólny algorytm rozwiązania problemu, wywoływana jest funkcja znajdująca uderzenie zbalansowane. W dalszej części algorytmu, podczas podziału zadania na mniejsze metodą "dziel i zwycieżaj", to punkty tego uderzenia nie będą uwzględnione w żadnym z podzadań. Skutkuje to redukcją rozmiaru podzadania o co najmniej $1$.
		Dla rozmiar zadania równego $0$ 
	\end{proof}	
	
	\begin{theorem}
		Algorytm ogólny rozwiązujący problem ma złożoność rzędu co najwyżej $O(n^{2} \log n)$.
	\end{theorem}
	
	\begin{proof}
	\end{proof}
	
	
\end{document}