\documentclass[10pt,a4paper]{article}
\usepackage[T1]{fontenc}
\usepackage[utf8]{inputenc}
\usepackage{graphicx}
\usepackage[polish]{babel}
\usepackage{hyperref}
\usepackage[paper=a4paper,margin=1in]{geometry}
\usepackage{tabularx} 
\usepackage{helvet}
\usepackage{amsfonts}
\usepackage{amsthm}
\usepackage{mathtools}
\usepackage{algpseudocode}
\usepackage{tikz}
\usepackage{tkz-graph}
\usepackage{listings,xcolor}
\renewcommand{\familydefault}{\sfdefault}
\setlength{\parindent}{0pt}
\newtheorem{theorem}{Twierdzenie}
\newtheorem{lemma}{Lemat}
\newtheorem{conculsion}{Wniosek}

\begin{document}
\begin{titlepage}
\newgeometry{top=1in,bottom=1in,right=1.5in,left=1.5in}
\begin{center}
{\fontsize{14}{12}\selectfont Wydział Matematyki i Nauk Informacyjnych Politechniki Warszawskiej}

\end{center}

\vspace{1cm}
\begin{center}
\includegraphics[width=0.3\textwidth]{images/logo.png}
\end{center}
\vspace{3cm}

\begin{center}
\textbf{{\fontsize{26}{12}\selectfont Metody sztucznej inteligencji 2}}

\vspace{2cm}
\textbf{{\fontsize{22}{12}\selectfont Konspekt projektu 1}}
\vspace{1cm}

\textbf{{\fontsize{13.5}{12}\selectfont Jacek Dziwulski, Mateusz Rymuszka}}

\vspace{6cm}
\textbf{{\fontsize{13.5}{12}\selectfont 24.02.2018}}
\end{center}  
\end{titlepage}

\section{Członkowie grupy}

W skład grupy wchodzą:

\begin{itemize}
\item Jacek Dziwulski
\item Mateusz Rymuszka
\end{itemize}

\section{Wybrany język programowania}

Projekt zrealizujemy w języku Python. Mamy w nim pewne doświadczenie, pozwalające na wykonanie zadania. Dodatkowo, dużym plusem jest możliwość łatwego rysowania wykresów. W implementacji zostanie wykorzystana biblioteka NumPy \cite{numpy}, pozwalająca wczytywać pliki csv, a także pozwalająca przechowywać dane w tablicach, na których można wygodnie operować. Ponadto, do wykonania wykresów posłuży biblioteka matplotlib \cite{matplot}.

\section{Krótki opis algorytmu}

W tym algorytmie zakładamy, że istnieje pewien zbiór uczący (zwany inaczej treningowym). Każda obserwacja należąca do tego zbioru, opisywana jest przez pewien wektor liczb rzeczywistych (zwanych zmiennymi objaśniającymi), opisujących cechy danej obserwacji. W przypadku gdy rozpatrujemy punkty z przestrzeni dwuwymiarowej, każdy taki wektor składa się z wartości $X_1$ oraz $X_2$. Ponadto, każda obserwacja w zbiorze treningowym ma przypisaną pewną wartość $Y$, zwaną zmienną objaśnianą\cite{python-ds-essentials}. 
\\~\\
Zadaniem jest przewidywanie wartości zmiennej objaśnianej, gdy na wejściu dostaniemy wektor zmiennych objaśniających\cite{python-ds-essentials}. 
\\~\\
W tym celu, należy znaleźć $k$ najbliższych sąsiadów (obserwacji) ze zbioru treningowego. Odległość pomiędzy punktami wyznaczana jest za pomocą wybranej metryki \cite{complete-guide-knn}. Najczęściej stosowana jest metryka euklidesowa. Ważnym parametrem jest parametr $k$, który określa liczbę poszukiwanych najbliższych sąsiadów. 
\\~\\
Wynik określany jest na podstawie wartości zmiennej objaśnianej znalezionych sąsiadów. Popularne strategie to na przykład średnia arytmetyczna z wartości zmiennej objaśnianej, mediana z tych wartości, czy też wybór najczęściej występującej wartości\cite{complete-guide-knn}.

\renewcommand{\refname}{\normalfont\selectfont\normalsize} 
\section{Literatura}
\begin{thebibliography}{9}
		
	\bibitem{numpy}
	NumPy 
	\textit{- array processing for numbers, strings, records, and object, http://www.numpy.org/}
	
	\bibitem{matplot}
	Matplotlib 
	\textit{- Python plotting package, https://matplotlib.org/}
	
	\bibitem{python-ds-essentials}
	Boschetti A., Massaron L.,
	\textit{Python Data Science Essentials}, Packt Publishing, 2015 
	
	\bibitem{complete-guide-knn}
	Zakka K., \textit{A Complete Guide to K-Nearest-Neighbors with Applications in Python and R, https://kevinzakka.github.io/2016/07/13/k-nearest-neighbor/} [dostęp 28.02.2018]

\end{thebibliography}

\end{document}